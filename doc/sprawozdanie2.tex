\documentclass[a4paper,10pt]{article}
\usepackage[utf8]{inputenc}
\usepackage[]{polski}
\usepackage{a4wide}
\usepackage{caption}
\usepackage{float}
\usepackage{amsthm}
\usepackage{graphicx}
\usepackage[T1]{fontenc}
\usepackage{listings}
\usepackage{multirow}
\usepackage{longtable}
\usepackage{algorithm}
%\usepackage{algorithmic}
\usepackage{algpseudocode}
\usepackage{enumerate}

\title{{\textbf{Grafy i Sieci}}\\[1ex]
       {\Large Sprawozdanie 2}\\[-1ex]
       {\Large Projekt - Graf częściowo spójne}}
\author{Krzysztof Opasiak, Adrian Brzozowski}

% Kropka po numerze paragrafu, podparagrafu itp.
\makeatletter
	\renewcommand\@seccntformat[1]{\csname the#1\endcsname.\quad}
	\renewcommand\numberline[1]{#1.\hskip0.7em}
\makeatother

% Kropka po numerze tablicy, rysunku i ustawienie czcionki dla etykiety.
\captionsetup{labelfont=sc,labelsep=period}

% Numeracja wzorów według paragrafu.
\renewcommand{\theequation}{\arabic{section}.\arabic{subsection}.\arabic{subsubsection}}

%\renewcommand{\thesubsection}{\alph{subsection}}
%\renewcommand{\thesubsubsection}{\Roman{subsubsection}}

% Zmiana nazwy figure "Rysunek" -> "Wykres"
\renewcommand{\figurename}{Wykres}

% Środowisko definition z numeracją
\newtheorem{definition}{Definicja}
\newtheorem{theorem}{Twierdzenie}


\date{Warszawa, \today r.}
\begin{document}
\maketitle

\section{Algorytm}
\subsection{Szczegółowy opis}

Pierwszym krokiem algorytmu wyznaczania składowych częściowo spójnych
grafu jest wyznaczenie jego składowych silnie spójnych.

\subsubsection{Wyznaczenie silnie spójnych składowych}
Do wyznaczenie składowych silnie spójnych wykorzystano algorytm
Kosaraju. Algorytm ten składa się z dwóch kroków. Pierwszym z nich
jest transpozycja grafu i przejrzenie wszystkich jego wierzchołków
używając algorytmu przeszukiwania w głąb i oznaczenie każdego
wierzchołka liczbą mówiącą w jakiej kolejności został on w pełni
przetworzony.

\begin{algorithm}
\caption{Oznaczenie wierzchołków}
\begin{algorithmic}
\State $t \leftarrow 0 $
\State $ordered\_verticles[] \leftarrow empty[]$

\Function{mark\_graph}{G}
\State $G2 \leftarrow transpose(G)$
\For{v in verticles(G2)}
  \If{v not explored}
    \State $mark\_node(G2, v)$
  \EndIf
\EndFor
\EndFunction

\Function{mark\_node}{G2, v}
  \State mark v as explored
  \For{e in edges\_from(G2, v)}
    \State $dest \leftarrow destination(e)$
    \If{dest not explored}
      \State $mark\_node(G2, dest)$
    \EndIf
  \EndFor
  \State $ordered\_verticles[t] \leftarrow v$
  \State $++t $
\EndFunction
\end{algorithmic}
\end{algorithm}

Kolejnym i ostatnim krokiem algorytmu Kosaraju jest przejrzenie grafu
w kolejności malejących oznaczeń i wyznaczenie składowych silnie
spójnych poprzez wyznaczenie algorytmem przeszukiwania w głąb
wszystkich wierzchołków osiągalnych z rozważanego wierzchołka.

\begin{algorithm}
\caption{Wyznaczanie silnie spójnych składowych}
\begin{algorithmic}
\State $current\_scc \leftarrow NULL $

\Function{find\_all\_scc}{G}
\State $scc[] \leftarrow empty[]$
\For{v decreasing in ordered\_verticles}
  \If{v not explored}
    \State $current\_scc \leftarrow new\_scc() $
    \State $mark\_scc(G, v)$
    \State $append(scc, current\_scc)$
  \EndIf
\EndFor\\
\Return scc
\EndFunction

\Function{mark\_scc}{G, v}
  \State mark v as explored
  \State $append(current\_scc, v)$
  \For{e in edges\_from(G, v)}
    \State $dest \leftarrow destination(e)$
    \If{dest not explored}
      \State $mark\_scc(G, dest)$
    \EndIf
  \EndFor
\EndFunction
\end{algorithmic}
\end{algorithm}

Po wykonaniu omówionych kroków otrzymamy podział grafu na składowe
silnie spójne.

\subsubsection{Utworzenie metagrafu}

Kolejnym krokiem algorytmu jest utworzenie metagrafu na podstawia
grafu G i jego podziału na składowe silnie spójne. W powstałym
metagrafie wierzchołki odpowiadają poszczególnym składowym silnie
spójnym. Ponadto krawędź w metagrafie miedzy wierzchołkami A i B
wystąpi jeśli dowolny z wierzchołków składowej A jest połączony z
dowolnym wierzchołkiem składowej B.

Zgodnie z algorytmem wyznaczania składowych silnie spójnych mamy
pewność, ze powstały w ten sposób metagraf jest acykliczny ponieważ
jeśli istniałby cykl to wierzchołki znalazłyby się w jednej składowej
silnie spójnej.

\subsubsection{Wyznaczenie składowych częściowo spójnych}

Składowe częściowo spójne można w prosty sposób uzyskać z metagrafu
poprzez wyznaczenie w nim wszystkich maksymalnych ścieżek. Ścieżki te
można wyznaczyć poprzez przeszukiwanie metagrafu i odnalezienie w nim
wszystkich wierzchołków, które nie posiadają krawędzi wchodzących. Dla
każdego z takich wierzchołków należy przeprowadzić przeszukiwanie w
głąb. Jeśli w trakcie tego przeszukiwania znajdziemy się w
wierzchołku, który nie ma krawędzi wychodzących to odnaleźliśmy
kolejną składową częściowo spójną.

\begin{algorithm}
\caption{Wyznaczanie składowych częściowo spójnych}
\begin{algorithmic}
\State $current\_wcc \leftarrow NULL $
\State $wcc[] \leftarrow empty[] $

\Function{find\_all\_wcc}{MG}
\For{v in verticles(MG)}
  \If{v doesn't have incomming edges}
    \State $current\_wcc \leftarrow new\_wcc() $
    \State $mark\_wcc(MG, v)$
  \EndIf
\EndFor\\
\Return wcc
\EndFunction

\Function{mark\_wcc}{G, v}
  \State $append(current\_wcc, v)$
  \State $out\_edges = edges\_from(G, v)$
  \If{out\_edges is empty}
    \State $append(wcc, clone(current\_wcc))$
    \State $remove(current\_wcc, v)$

    \Return
  \EndIf

  \For{e in out\_edges}
    \State $dest \leftarrow destination(e)$
    \State $mark\_wcc(G, dest)$
  \EndFor

\EndFunction
\end{algorithmic}
\end{algorithm}


Omówiony powyżej krok stanowi już ostatnią część algorytmu. Po jego
wykonaniu uzyskujemy wszystkie częściowo spójne składowe co powoduje
zakończenie algorytmu.

\subsection{Wykazanie poprawności}
\subsection{Złożoność czasowa}
Algorytm składa się z kroków o następującej złożoności:


\subsubsection{Algorytm Kosaraju}
\begin{center}
{\large\bf O(m + n)}
\end{center}

Algorytm ten wykonuje dwa pełne przeglądy grafy G używając metody
przeszukiwania w głąb. Metoda ta gwarantuje odwiedzenie wszystkich
wierzchołków jeden raz.

\subsubsection{Utworzenie metagrafu}
\begin{center}
{\large\bf O(m + n)}
\end{center}

W tym kroku algorytmu musimy przejrzeć ponownie wszystkie wierzchołki i
krawędzie aby utworzyć krawędzie metagrafu.

\subsubsection{Wyszukanie najdłuższych ścieżek}
\begin{center}
{\large\bf O(m2 + n2) gdzie m2 $\leq$ m, n2 $\leq$ n}
\end{center}

W tym kroku musimy odwiedzić wszystkie wierzchołki metagrafu i przejrzeć
wszystkie jego krawędzie. Koszt takiej operacji jest liniowy względem
liczby wierzchołków (i krawędzi) metagrafu, których jest tyle ile
składowych silnie spójnych więc nie więcej niż wierzchołków w grafie
wejściowym.

\subsubsection{Ogólna złożoność}
\begin{center}
{\large\bf O(m + n)}
\end{center}

Czynnikiem o najwyższym koszcie jest w tym algorytmie wyznaczenie
składowych silnie spójnych, zatem złożoność całego algorytmu jest
równa temu najwyższemu czynnikowi.

\subsection{Możliwe zastosowania}
\subsection{Dowód zadanego twierdzenia}
\section{Program}
\subsection{Podstawowe założenia}
\begin{itemize}
\item Program pobiera dane wejściowe poprzez standardowe wejście
\item Dane wejściowe programu mają następujący format:
  \begin{itemize}
  \item Linia zawierająca liczbę całkowitą N stanowiącą liczbę wierzchołków grafu.
  \item N linii zawierających liczby całkowite, które stanowią numery
    wierzchołków do których wychodzą krawędzie z danego wierzchołka
  \end{itemize}
\item Jedna linia jest nie dłuższa niż 4096 znaków (dla wygody wczytywania)
\item Wyjście programu będzie stanowić M linii, z których każda będzie
  zawierała liczby naturalne stanowiące numery wierzchołków, które
  wchodzą w skład danej składowej częściowo spójnej (jedna składowa,
  jedna linia).
\item Program zostanie zaimplementowany w języku C lub C++
\end{itemize}

\subsection{Projekt testów}

Testy programu będą się składać z dwóch etapów. Pierwszy etap będą to
deterministyczne testy na kilku przygotowanych uprzednio grafach, dla
których podział na składowe częściowo spójne został wykonany inny
programem lub ręcznie. W trakcie tego etapu sprawdzamy czy
wygenerowane przez program składowe częściowo spójne pokrywają się z
tymi wyznaczonymi inną metodą. Drugi etap testów będą stanowiły testy
niedeterministyczne. Dla kilkunastu losowych grafów zostanie
uruchomiony algorytm, a następnie sprawdzone zostanie czy wygenerowane
składowe częściowo spójne spełniają warunek bycia poprawną składową
częściowo spójną.

\end{document}

