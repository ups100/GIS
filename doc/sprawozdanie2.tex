\documentclass[a4paper,10pt]{article}
\usepackage[utf8]{inputenc}
\usepackage[]{polski}
\usepackage{a4wide}
\usepackage{caption}
\usepackage{float}
\usepackage{amsthm}
\usepackage{graphicx}
\usepackage[T1]{fontenc}
\usepackage{listings}
\usepackage{multirow}
\usepackage{longtable}
\usepackage{algorithm}
%\usepackage{algorithmic}
\usepackage{algpseudocode}
\usepackage{enumerate}

\newtheorem{twr}{Twierdzenie}
\newtheorem{lem}[twr]{Lemat}

\title{{\textbf{Grafy i Sieci}}\\[1ex]
       {\Large Sprawozdanie 2}\\[-1ex]
       {\Large Projekt - Graf częściowo spójne}}
\author{Krzysztof Opasiak, Adrian Brzozowski}

% Kropka po numerze paragrafu, podparagrafu itp.
\makeatletter
	\renewcommand\@seccntformat[1]{\csname the#1\endcsname.\quad}
	\renewcommand\numberline[1]{#1.\hskip0.7em}
\makeatother

% Kropka po numerze tablicy, rysunku i ustawienie czcionki dla etykiety.
\captionsetup{labelfont=sc,labelsep=period}

% Numeracja wzorów według paragrafu.
\renewcommand{\theequation}{\arabic{section}.\arabic{subsection}.\arabic{subsubsection}}

%\renewcommand{\thesubsection}{\alph{subsection}}
%\renewcommand{\thesubsubsection}{\Roman{subsubsection}}

% Zmiana nazwy figure "Rysunek" -> "Wykres"
\renewcommand{\figurename}{Wykres}

% Środowisko definition z numeracją
\newtheorem{definition}{Definicja}
\newtheorem{theorem}{Twierdzenie}


\date{Warszawa, \today r.}
\begin{document}
\maketitle

\section{Algorytm}
\subsection{Szczegółowy opis}

Pierwszym krokiem algorytmu wyznaczania składowych częściowo spójnych
grafu jest wyznaczenie jego składowych silnie spójnych.

\subsubsection{Wyznaczenie silnie spójnych składowych}
Do wyznaczenie składowych silnie spójnych wykorzystano algorytm
Kosaraju. Algorytm ten składa się z dwóch kroków. Pierwszym z nich
jest transpozycja grafu i przejrzenie wszystkich jego wierzchołków
używając algorytmu przeszukiwania w głąb i oznaczenie każdego
wierzchołka liczbą mówiącą w jakiej kolejności został on w pełni
przetworzony.

\begin{algorithm}
\caption{Oznaczenie wierzchołków}
\begin{algorithmic}
\State $t \leftarrow 0 $
\State $ordered\_verticles[] \leftarrow empty[]$

\Function{mark\_graph}{G}
\State $G2 \leftarrow transpose(G)$
\For{v in verticles(G2)}
  \If{v not explored}
    \State $mark\_node(G2, v)$
  \EndIf
\EndFor
\EndFunction

\Function{mark\_node}{G2, v}
  \State mark v as explored
  \For{e in edges\_from(G2, v)}
    \State $dest \leftarrow destination(e)$
    \If{dest not explored}
      \State $mark\_node(G2, dest)$
    \EndIf
  \EndFor
  \State $ordered\_verticles[t] \leftarrow v$
  \State $++t $
\EndFunction
\end{algorithmic}
\end{algorithm}

Kolejnym i ostatnim krokiem algorytmu Kosaraju jest przejrzenie grafu
w kolejności malejących oznaczeń i wyznaczenie składowych silnie
spójnych poprzez wyznaczenie algorytmem przeszukiwania w głąb
wszystkich wierzchołków osiągalnych z rozważanego wierzchołka.

\begin{algorithm}
\caption{Wyznaczanie silnie spójnych składowych}
\begin{algorithmic}
\State $current\_scc \leftarrow NULL $

\Function{find\_all\_scc}{G}
\State $scc[] \leftarrow empty[]$
\For{v decreasing in ordered\_verticles}
  \If{v not explored}
    \State $current\_scc \leftarrow new\_scc() $
    \State $mark\_scc(G, v)$
    \State $append(scc, current\_scc)$
  \EndIf
\EndFor\\
\Return scc
\EndFunction

\Function{mark\_scc}{G, v}
  \State mark v as explored
  \State $append(current\_scc, v)$
  \For{e in edges\_from(G, v)}
    \State $dest \leftarrow destination(e)$
    \If{dest not explored}
      \State $mark\_scc(G, dest)$
    \EndIf
  \EndFor
\EndFunction
\end{algorithmic}
\end{algorithm}

Po wykonaniu omówionych kroków otrzymamy podział grafu na składowe
silnie spójne.

\subsubsection{Utworzenie metagrafu}

Kolejnym krokiem algorytmu jest utworzenie metagrafu na podstawia
grafu G i jego podziału na składowe silnie spójne. W powstałym
metagrafie wierzchołki odpowiadają poszczególnym składowym silnie
spójnym. Ponadto krawędź w metagrafie miedzy wierzchołkami A i B
wystąpi jeśli dowolny z wierzchołków składowej A jest połączony z
dowolnym wierzchołkiem składowej B.

Zgodnie z algorytmem wyznaczania składowych silnie spójnych mamy
pewność, ze powstały w ten sposób metagraf jest acykliczny ponieważ
jeśli istniałby cykl to wierzchołki znalazłyby się w jednej składowej
silnie spójnej.

\subsubsection{Wyznaczenie składowych częściowo spójnych}

Składowe częściowo spójne można w prosty sposób uzyskać z metagrafu
poprzez wyznaczenie w nim wszystkich maksymalnych ścieżek. Ścieżki te
można wyznaczyć poprzez przeszukiwanie metagrafu i odnalezienie w nim
wszystkich wierzchołków, które nie posiadają krawędzi wchodzących. Dla
każdego z takich wierzchołków należy przeprowadzić przeszukiwanie w
głąb. Jeśli w trakcie tego przeszukiwania znajdziemy się w
wierzchołku, który nie ma krawędzi wychodzących to odnaleźliśmy
kolejną składową częściowo spójną.

\begin{algorithm}
\caption{Wyznaczanie składowych częściowo spójnych}
\begin{algorithmic}
\State $current\_wcc \leftarrow NULL $
\State $wcc[] \leftarrow empty[] $

\Function{find\_all\_wcc}{MG}
\For{v in verticles(MG)}
  \If{v doesn't have incomming edges}
    \State $current\_wcc \leftarrow new\_wcc() $
    \State $mark\_wcc(MG, v)$
  \EndIf
\EndFor\\
\Return wcc
\EndFunction

\Function{mark\_wcc}{G, v}
  \State $append(current\_wcc, v)$
  \State $out\_edges = edges\_from(G, v)$
  \If{out\_edges is empty}
    \State $append(wcc, clone(current\_wcc))$
    \State $remove(current\_wcc, v)$

    \Return
  \EndIf

  \For{e in out\_edges}
    \State $dest \leftarrow destination(e)$
    \State $mark\_wcc(G, dest)$
  \EndFor

\EndFunction
\end{algorithmic}
\end{algorithm}


Omówiony powyżej krok stanowi już ostatnią część algorytmu. Po jego
wykonaniu uzyskujemy wszystkie częściowo spójne składowe co powoduje
zakończenie algorytmu.

\subsection{Wykazanie poprawności}

Poprawność algorytmu Kosaraju jak i sam algorytm został opisany w
\cite[rozdz. 23.5]{Cormen}. Ponadto opis ten zawiera dowód, iż
metagraf skonstrułowany po wyznaczeniu składowych silnie spójnych
będzie grafem acyklicznym. Brakującym elementem do udowodnienia
poprawności całego algorytmu jest wykazanie, że w ostatnim kroku
algorytmu zostaną wygenerowane wszystkie składowe częściowo spójne.

Dowód poprawności algorytmu rozpoczniemy od następującej obserwacji:

\begin{lem}
\label{silnie czesciowo}
Każda składowa silnie spójna jest również częściowo spójna.
\end{lem}

\begin{proof}
  Dowód tego lematu jest trywialny. Aby podgraf nie był częściowo
  spójny musi istnieć w nim para wierzchołków dla której nie ma ścieżki
  w dowolną stronę. Załóżmy więc, że dla pewnej pary wierzchołków u i v
  ze składowej spójnej nie istnieje ścieżka z u do v oraz z v do
  u. Dochodzimy automatycznie do sprzeczności, gdyż w składowej
  spójnej musi dla każdej pary wierzchołków istnieć ścieżka z u do v
  jak i z v do u.
\end{proof}

\begin{lem}
\label{sciezka}
  Graf jest częściowo spójny, jeśli zawiera skierowaną ścieżkę
  przechodzącą przez co najmniej jeden wierzchołek każdej z jego
  składowej silnie spójnej.
\end{lem}

\begin{proof}
  Na podstawie \label{silnie czesciowo} wiemy, że każda ze składowych
  silnie spójnych jest również częściowo spójna. Dodatkowym założeniem
  wprowadzanym w \label{sciezka} jest istnienie skierowanej ścieżki
  zawierającej co najmniej jeden wierzchołek z każdej silnie spójnej
  składowej. Weźmy zatem dwa dowolne wierzchołki u i v. Wierzchołki te
  mogą być wzajemnie w trzech różnych położeniach:
  \begin{description}
  \item[u i v są w tej samej silnie spójnej składowej.] W silnie
    spójnej składowej istnieje ścieżka z u do v jak i z v do u co nie
    potrzebuje dalszego dowodu.
  \item[u i v są w różnych składowych i znajdują się na ścieżce
    łączącej składowe.] Skro u oraz v znajdują się na pewnej
    skierowanej ścieżce co automatycznie oznacza, że ścieżka pomiędzy
    tymi wierzchołkami istnieje.
  \item[u i v są w różnych składowych i nie znajdują się na ścieżce
    łączącej składowe.] Wierzchołki te nie znajdują się na wspólnej
    ścieżce ale składowej do której należy wierzchołek u znajduje się
    pewien wierzchołek w, który leży na tej ścieżce. Analogicznie w
    składowej w której znajduje się wierzchołek v jest pewien
    wierzchołek x, który leży na ścieżce. Zatem mamy pewne wierzchołki
    x oraz w zawierają się w skierowanej ścieżce łączącej wszystkie
    składowe spójne. Ponieważ w znajduje się w tej samej silnie
    spójnej składowej co u to istnieje ścieżka prowadząca od w do u i
    odwrotnie. Analogicznie dla wierzchołków v i x. Oznacza to, że
    możemy utworzyć ścieżkę z u do v, która będzie prowadziła przez
    wierzchołki w i x.
  \end{description}
  Ponieważ wyczerpaliśmy już wszystkie możliwości i pokazaliśmy, że w
  każdej z nich można utworzyć skierowaną ścieżkę pomiędzy dowolnymi
  dwoma wierzchołkami (w co najmniej jedną ze stron) udowodniliśmy, że
  graf ten jest częściowo spójny.
\end{proof}

Powyższe lematy oraz dowody pokazują, że graf częściowo spójny może
zostać uzyskany poprzez połączenie skierowaną ścieżką dowolnych
wierzchołków z składowych silnie spójnych. Zadanie natomiast dotyczy
znalezienia maksymalnych takich podgrafów. W przyjętym metamodelu
każda składowa silnie spójna jest reprezentowana przez jeden
wierzchołek zatem należy jeszcze wykazać, w jaki sposób można
wyznaczyć wszystkie najdłuższe ścieżki w grafie acyklicznym.

\begin{lem}
\label{sciezki_acykl}
W grafie acyklicznym można wyznaczyć wszystkie najdłuższe ścieżki
poprzez przeszukiwanie grafu w głąb rozpoczynając przeszukiwanie od
wszystkich wierzchołków, które nie posiadają krawędzi wchodzących.
\end{lem}

\begin{proof}
  Z własności grafu acyklicznego wynika, że w każdym takim grafie musi
  istnieć co najmniej jeden wierzchołek nie posiadających krawędzi
  wchodzących jak i co najmniej jeden wierzchołek nie posiadający
  krawędzi wychodzących. Ponadto wiadomym jest, że ze względu na
  acykliczność grafu podczas przeszukiwania grafu w głąb (od pewnego
  punktu) nigdy nie trafimy do wierzchołka, który już został
  odwiedzony. Natomiast ze względu na przeszukiwanie grafu w głąb mamy
  gwarancję odwiedzenia wszystkich wierzchołków osiągalnych z
  wybranego wierzchołka startowego. Najdłuższe możliwe ścieżki
  uzyskujemy gdy rozpoczynając przeszukiwanie od wierzchołka v, który
  nie ma krawędzi wchodzących dojdziemy do pewnego wierzchołka u, z
  którego nie możemy już iść dalej. Ponieważ v nie ma krawędzi
  wchodzących, a u wychodzących to nie jest możliwe, aby znaleziona
  ścieżka w całości znajdowała się w innej ścieżce o większej
  długości.
\end{proof}

Wykorzystując powyższe lematy udowodniono, że zastosowany algorytm
jest poprawny i znajdzie on wszystkie maksymalne częściowo spójne
podgrafy.

\subsection{Złożoność czasowa}
Algorytm składa się z kroków o następującej złożoności:


\subsubsection{Algorytm Kosaraju}
\begin{center}
{\large\bf O(m + n)}
\end{center}

Algorytm ten wykonuje dwa pełne przeglądy grafy G używając metody
przeszukiwania w głąb. Metoda ta gwarantuje odwiedzenie wszystkich
wierzchołków jeden raz.

\subsubsection{Utworzenie metagrafu}
\begin{center}
{\large\bf O(m + n)}
\end{center}

W tym kroku algorytmu musimy przejrzeć ponownie wszystkie wierzchołki i
krawędzie aby utworzyć krawędzie metagrafu.

\subsubsection{Wyszukanie najdłuższych ścieżek}
\begin{center}
{\large\bf O(m2 + n2) gdzie m2 $\leq$ m, n2 $\leq$ n}
\end{center}

W tym kroku musimy odwiedzić wszystkie wierzchołki metagrafu i przejrzeć
wszystkie jego krawędzie. Koszt takiej operacji jest liniowy względem
liczby wierzchołków (i krawędzi) metagrafu, których jest tyle ile
składowych silnie spójnych więc nie więcej niż wierzchołków w grafie
wejściowym.

\subsubsection{Ogólna złożoność}
\begin{center}
{\large\bf O(m + n)}
\end{center}

Czynnikiem o najwyższym koszcie jest w tym algorytmie wyznaczenie
składowych silnie spójnych, zatem złożoność całego algorytmu jest
równa temu najwyższemu czynnikowi.

\subsection{Możliwe zastosowania}

Podział na składowe częściowo spójne możemy wykorzystać do szybkiego
określenie czy graf może posiadać cykl Hamiltona. Ponieważ jest to
problem NP-zupełny to wykorzystując zależność, że każdy graf
Hamiltonowski musi być jedną składową częściowo spójną możemy w czasie
liniowym określić, że jeśli graf ma więcej niż jedną składową
częściowo spójną to na pewno nie ma cyklu Hamiltona (Oczywiście
implikacja w drugą stronę nie jest prawdziwa).

Ponadto możemy użyć tego podziału do określenia jaką największą liczbę
wierzchołków możemy odwiedzić startując z jednego dowolnego
punktu. Może to być przydatne we wszystkich problemach związanych z
problemem komiwojażera. Pozwala to np bardzo prosto powiedzieć, że nie
da się odwiedzić wszystkich wierzchołków, a także wskazać którą
podgrupę wierzchołków najbardziej opłaca się rozważać aby odwiedzić
ich jak najwięcej.

\subsection{Dowód zadanego twierdzenia}

\begin{twr}
  Graf G = (V, E) jest częściowo spójny wtedy i tylko wtedy, gdy
  zawiera skierowaną ścieżkę zawierającą wszystkie wierzchołki ze
  zbioru V.
\end{twr}

\begin{proof}
  Dowód w lewo jest trywialny. Jeśli istnieje ścieżka, która zawiera
  wszystkie wierzchołki grafu to oznacza, że dowolna para wierzchołków
  tego grafu znajduje się na tej ścieżce. Implikuje to, że każda para
  wierzchołków jest ze sobą połączona i osiągalna w co najmniej jedna
  ze stron. Powyższe stanowi definicję grafu częściowo spójnego zatem
  dowód w lewo zakończony.

  Dowód w prawo. Załóżmy, że istnieje w G taki wierzchołek v, że nie
  można go dodać, do ścieżki przechodzącej przez wszystkie
  wierzchołki. Oznacza to jedna z poniższych sytuacji:

  \begin{description}
  \item[v jest wierzchołkiem swobodnym] Jeśli nie posiada on krawędzi
    wchodzących jak i wychodzących to owszem nie można poprowadzić
    przez niego ścieżki, jednak powoduje to również sprzeczność bo G
    nie może być częściowo spójny
  \item[v nie jest jedynym wierzchołkiem, który posiada jedynie
    krawędzie wychodzące] Oznacza to sytuacje, w której w grafie g
    istnieje para wierzchołków u i v, które posiadają jedynie
    krawędzie wychodzące. Doprowadza to do sprzeczności z założeniem,
    że graf jest częściowo spójny ponieważ między u i v nie da się
    wyznaczyć żadnej ścieżki.
    \item[v nie jest jedynym wierzchołkiem, który posiada jedynie
    krawędzie wchodzące] W sposób analogiczne sytuacja doprowadza nas do sprzeczności
  \end{description}

  We wszystkich pozostałych przypadkach odpowiednią ścieżkę da się w
  sposób oczywisty poprowadzić również przez v.
\end{proof}

\section{Program}
\subsection{Podstawowe założenia}
\begin{itemize}
\item Program pobiera dane wejściowe poprzez standardowe wejście
\item Dane wejściowe programu mają następujący format:
  \begin{itemize}
  \item Linia zawierająca liczbę całkowitą N stanowiącą liczbę wierzchołków grafu.
  \item N linii zawierających liczby całkowite, które stanowią numery
    wierzchołków do których wychodzą krawędzie z danego wierzchołka
  \end{itemize}
\item Jedna linia jest nie dłuższa niż 4096 znaków (dla wygody wczytywania)
\item Wyjście programu będzie stanowić M linii, z których każda będzie
  zawierała liczby naturalne stanowiące numery wierzchołków, które
  wchodzą w skład danej składowej częściowo spójnej (jedna składowa,
  jedna linia).
\item Program zostanie zaimplementowany w języku C lub C++
\end{itemize}

\subsection{Projekt testów}

Testy programu będą się składać z dwóch etapów. Pierwszy etap będą to
deterministyczne testy na kilku przygotowanych uprzednio grafach, dla
których podział na składowe częściowo spójne został wykonany inny
programem lub ręcznie. W trakcie tego etapu sprawdzamy czy
wygenerowane przez program składowe częściowo spójne pokrywają się z
tymi wyznaczonymi inną metodą. Drugi etap testów będą stanowiły testy
niedeterministyczne. Dla kilkunastu losowych grafów zostanie
uruchomiony algorytm, a następnie sprawdzone zostanie czy wygenerowane
składowe częściowo spójne spełniają warunek bycia poprawną składową
częściowo spójną.

\end{document}

\begin{thebibliography}{9}
\bibitem{Cormen} T.H. Cormen, C.E. Leiserson, R.L. Rivest, \emph{Wprowadzenie do algorytmów}, Wydawnictwo Naukowo-Techniczne, Warszawa 2001.
\end{thebibliography}
\end{document}

