\documentclass[a4paper,10pt]{article}
\usepackage[utf8]{inputenc}
\usepackage[]{polski}
\usepackage{a4wide}
\usepackage{caption}
\usepackage{float}
\usepackage{amsthm}
\usepackage{graphicx}
\usepackage[T1]{fontenc}
\usepackage{listings}
\usepackage{multirow}
\usepackage{longtable}

\title{{\textbf{Grafy i Sieci}}\\[1ex]
       {\Large Sprawozdanie 1}\\[-1ex]
       {\Large Projekt - Graf częściowo spójne}}
\author{Krzysztof Opasiak, Adrian Brzozowski}

% Kropka po numerze paragrafu, podparagrafu itp.
\makeatletter
	\renewcommand\@seccntformat[1]{\csname the#1\endcsname.\quad}
	\renewcommand\numberline[1]{#1.\hskip0.7em}
\makeatother

% Kropka po numerze tablicy, rysunku i ustawienie czcionki dla etykiety.
\captionsetup{labelfont=sc,labelsep=period}

% Numeracja wzorów według paragrafu.
\renewcommand{\theequation}{\arabic{section}.\arabic{subsection}.\arabic{subsubsection}}

\renewcommand{\thesubsection}{\alph{subsection}}
\renewcommand{\thesubsubsection}{\Roman{subsubsection}}

% Zmiana nazwy figure "Rysunek" -> "Wykres"
\renewcommand{\figurename}{Wykres}

% Środowisko definition z numeracją
\newtheorem{definition}{Definicja}
\newtheorem{theorem}{Twierdzenie}


\date{Warszawa, \today r.}
\begin{document}
\maketitle

\section{Opis merytoryczny zadania}

Zadanie polega na opracowaniu i zaimplementowaniu algorytmu
wyznaczania składowych częściowo spójnych w zadanym grafie
skierowanym. Wejście programu stanowić będzie opis tekstowy grafu w
następującym formacie:

\begin{itemize}
\item Linia zawierająca liczbę całkowitą N stanowiącą liczbę wierzchołków grafu.
\item N linii zawierających liczby całkowite, które stanowią numery
  wierzchołków do których wychodzą krawędzie z danego wierzchołka
\end{itemize}

Dla wygody wczytywania i prostoty implementacji zakładamy, ze
pojedyncza linia mieści się w buforze o rozmiarze 4096 bajtów.

Wyjście programu będzie stanowić M linii, z których każda będzie
zawierała liczby naturalne stanowiące numery wierzchołków, które
wchodzą w skład danej składowej częściowo spójnej (jedna składowa,
jedna linia).

\section{Opis wykorzystywanych algorytmów}

Jako podstawę do rozwiązania tego problemu planujemy wykorzystać
algorytm {\it STRONGLY-CONNECTED-COMPONENTS} opisany w
\cite[rozdz. 23.5]{Cormen}. Jest to algorytm, który wykorzystuje przeszukiwanie
grafu w głąb. Po przeszukaniu grafu oraz grafu transponowanego w głąb
otrzymujemy podział grafu na składowe silnie spójne. Oznacza to, że z
dowolnego wierzchołka v w takiej składowej istnieje ścieżka do
dowolnego wierzchołka u oraz powrotna.

Dzięki zastosowaniu tego algorytmu możemy utworzyć graf G', w którym
wierzchołki będą stanowiły składowe silnie spójne. Krawędź w takim
grafie będzie łączyła dwa wierzchołki, jeśli w grafie G istniała
krawędź pomiędzy dowolnymi dwoma wierzchołkami łącząca te dwie składowe.

Istotną własnością zbudowanego w ten sposób grafu jest jego
acykliczność. Graf ten nie może posiadać cyklu, ponieważ jego
istnienie implikowałoby istnienie ścieżki z dowolnego wierzchołka
jednej składowej do dowolnego wierzchołka innej składowej (składowych)
co z kolei oznaczałoby, że cały ten cykl stanowi jedną składową silnie
spójną.

Ostatnim krokiem algorytmu będzie wyszukanie wszystkich najdłuższych
ścieżek w takim grafie.

\begin{thebibliography}{9}
\bibitem{Cormen} T.H. Cormen, C.E. Leiserson, R.L. Rivest, \emph{Wprowadzenie do algorytmów}, Wydawnictwo Naukowo-Techniczne, Warszawa 2001.
\end{thebibliography}
\end{document}

